\documentclass{article}
\usepackage{graphicx}
\usepackage{url}
\usepackage[left=3cm, right=3cm]{geometry}
\title{Modelling and Simulation of Gene Regulatory Networks}
\author{Ilkut Kutlar - 1621364}
\date{October 2018}

\begin{document}
	\maketitle
	
	\section{Problem Statement}
	
	\par Synthetic Biology is a relatively new and rapidly growing field focusing on biological constructs not found in nature. One specific area of focus is Genetic Regulatory Networks (GRNs), where the aim of researchers is to create useful genetic circuits capable of producing desired amounts of proteins, such as the Repressilator \cite{repressilator}, a novel circuit which produces three proteins at amounts which oscillate with time.
	% This has useful applications such as 
	\par Since Synthetic Biology deals with new circuits, there tends to be a design stage before the network is put together in a real organism. Computer Aided Design (CAD) can help accomplish this task.
	\par Some existing CAD and simulation software include COPASI\cite{copasi}, a tool for simulating biologcal reactions (not only related to GRNs). Another tool, GenoCAD\cite{genocad}, is a web-based software allowing design of genetic circuits from a database of parts, export it to some popular formats (such as SBML), and simulate the design using COPASI's API. TinkerCell\cite{tinkercell} is a similar desktop-based software, allowing drag-and-drop gene circuit design and simulation.
	\par This project will include features to design and simulate gene circuits, just as simulating tools do. One feature not offered by these software is reverse engineering of circuits, which this project will implement. This feature will allow the user to specify a collection of constraints and properties (such as the required concentration of a certain product) and create a gene circuit having all the properties and obeying all the constraints. This can save time for researchers.
	%TODO: Why is this feature so useful? Elaborate more on this
	
	% - (Some possible extensions:
	% - IDE functionality
	% - Calculation of values from sequences)
	
	
	% (Extra constraints could include: Do it in this organism, do it using only these parts, do it using so many parts, etc.)
	
	\section{Objectives}
		\par The software will let the user choose biological parts from a catalogue (which includes user generated parts as well as those fetched from a host of public databases), combine them (similar to building an electric circuit) and set the details of the relationship between them (such as the equations describing their regulatory relation). Software will then be able to simulate the circuit to observe the concentration of products over time. The original feature of the project will allow the user to state the desired amount of products and let the software reverse engineer a circuit capable of producing them. These objectives can be expressed more specifically as a list of main tasks, and a set of subtasks that needs to be completed for the main task to be complete:
		% TODO: I don't like the wording here
		
		\begin{enumerate}
			\item Should allow adding parts (as well as edit and delete them) to the network and specify their regulatory relations. Should allow the user to specify necessary network parameters.
			\begin{enumerate}
				\item Software should have a programmatic model of every supported part and its various parameters. This also includes non-trivial substances (which are not proteins or mRNA) which affect regulation, such lactose, as well as operons and other constructs which may be of importance to the process of regulation.
				\item Should keep track of the relations between each part (e.g. Which gene regulates a specific gene using which equation).
				\item Should store a programmatic model of the whole network, including parts, relations and global values. This subtask will build the foundation for the programmatic architecture of the project.
			\end{enumerate}
		
			\item Must be able to simulate the circuit.
			\begin{enumerate}
				\item Must be able to convert the programmatic model of the network into a series of mathematical equations which can be fed into an equation solver.
				\item Must solve the created equations and produce results (concentrations of products) at given intervals over a given period of time.
				\item Must be able to visualise the results of the simulation (e.g. using graphs).
			\end{enumerate}
			
			\item Must allow the user to specify a set of desired properties for the circuit to have (e.g. The amount produced of a certain protein should be above a certain number and the circuit should only use a given set of parts) and subsequently build a network which have these features.
			\begin{enumerate}
				\item Must convert the constraints given by the user into a set of mathematical constraints which the internal logic can work with.
				\item Needs to narrow down all possible circuit combinations to find the right one(s). 
			\end{enumerate}
	
			\item Must offer a user friendly UI.
			\begin{enumerate}
				\item Must visualise the circuit and its parts on the screen.
				\item Must allow drag-and-drop style interaction when adding and moving parts and defining relations between them.
				\item Must offer easy and intuitive access to parameters and values (such as clicking on a gene to access its associated parameters).
			\end{enumerate}
		
			\item It should be able to fetch biological parts (such as promoters, coding regions, etc.) and models (such as the Repressilator) from public databases (such as the BioModels\cite{biomodels} database, Registry of Standard Biological Parts \cite{rsbp}). Also needs to allow manuallly adding parts.
			\begin{enumerate}
				\item For some databases, needs to be able to scrape webpages.
				\item For some databases, needs to download machine-readable files and parse them.
				\item Needs to allow creation of parts from values entered by the user.
			\end{enumerate}
			
			\item Can import parts and models from a host of popular formats (such as SMBL) and export to them.
			\begin{enumerate}
				\item Needs to be able to parse these formats.
				\item Needs to be able to convert the internal model used by this project to other popular formats.
			\end{enumerate}
	
		\end{enumerate}
	
	\subsection{Accuracy of the model}
	\par To be able to computationally represent a real Gene Regulatory Network (GRN), the software will have to adopt a mathematical model for GRNs. Researches have come up with various models of varying accuracy, ranging from boolean networks (which abstract some regulatory parameters) 
	% TODO: [CITE]
	to more accurate stochastic models (which aim to be accurate and include random noise).
	%TODO: [CITE]
	Furthermore, within each model, it is possible to abstract some factors which would have some effect on the process in real life. I am going to use Ordinary Differential Equations (ODEs) 
	%TODO: [CITE]
	to model the change in the concentration of mRNAs and proteins and model the gene regulation using Hill equation
	%TODO: [CITE]
	and take Hill coefficient to be 1. Finally, I will be considering the transcription rate, translation rate and mRNA and protein degradation for regulating transcription.
	
	\par However, it is possible that my chosen model may be too detailed and the time to implement it may cause me to miss some important milestones, or may be too abstract and not challenging enough. To avoid either problems, I have chosen a number of important dates (See the timetable section) on which I will consider my progress and decide whether I should revise my chosen model to make it more or less detailed.
	
	
	\section{Methods}
	
	\subsection{Software Methodology}
	%TODO This can be expanded
	\par I will be using a Scrum methodology, and in each sprint I will implement a new objective.

		
	\subsection{Version Management}
	\par For version management, I will be using Git, an open source software and will be pushing my changes to a remote private repository hosted at GitHub on my student account.
	
	\subsection{Evaluation \& Testing}
	\par I will be writing appropriate unit tests for each of the objectives apart from the ones related to a user friendly UI, as those cannot be tested with a unit test.
	\par To evaluate the circuit design and simulation features as a whole, I will build a network for which simulation results have been made available by other authors. The network will be using the same initial values, and the test will check whether my software produces the same results. 
	\par To evaluate the reverse engineering feature, I will input a set of constraints and let the software produce a circuit. Consequently, I will simulate the circuit in COPASI\cite{copasi}, an open source biological simulation software, and manually check whether the results obey the given constraints.
	
	\subsection{Timetable}
	%TODO: Write something here!
	%TODO: Final report goes on for the whole duration
	%TODO: Also include dependencies!
	\includegraphics[height=100pt]{timetable}
	
	\section{Resources}
	\begin{enumerate}
		\item The software will be written in Python.
		\item The simulation feature will require libraries capable of handling mathematical operations. For that, I will be using NumPy \cite{numpy} and SciPy \cite{scipy}.
		\item For the visualisation of the simulation results, a graph plotting library will be required. For this, I will use matplotlib \cite{matplotlib}.
		\item Git and GitHub will be used for version control and backing up the software.
		\item The data required (for biological parts and models) will be fetched from public databases such as the BioModels \cite{biomodels} and the Registry of Standard Biological Parts. \cite{rsbp}
	\end{enumerate}

	\section{Risks}
	\paragraph{IT failure} I will be regularly pushing my local commits on Git to GitHub, a remote repository. Therefore, if my personal computer fails I can carry on working using departmental computers.
	\paragraph{Underestimation of the time required} In the case that I have not been able to complete some required tasks on time, leading to the project getting derailed, I will focus on the most essential features and not implement some non-essential features, such as having a user friendly UI.
	\paragraph{Unexpected problem preventing me from working for a period of time} To avoid such problems, I have overestimated the time required to complete tasks in my timetable.
	
	\section{Ethical Considerations}
	\begin{itemize}
		\item All software and services I plan to use (except GitHub) are free and open source and their licences do not restrict usages for a university dissertation project.
		\item GitHub normally charges a monthly fee for a private repository. However, as I have a student account, I am able to freely use a private repository.
		\item All data required for the software is publicly available.
	\end{itemize}

	\bibliographystyle{plain}
	\bibliography{specification}
\end{document}