\documentclass{article}
\usepackage[margin=1.2in]{geometry}
\usepackage{url}
\usepackage{appendix}
\author{Ilkut Kutlar - u1621364}
\title{CS310 - Progress Report}
\usepackage{graphicx}
\begin{document}
	\maketitle
	\tableofcontents
	
	\newpage
	
	\section{Introduction}
	
	Synthetic Biology is a relatively new field concerned with creating new biological constructs not found in the nature with the aim of serving a useful purpose. The field involves an engineering process and therefore a researcher may need to go through a design stage before actually implementing the construct in a real organism. This project aims to make a CAD and simulation software for Gene Regulatory Networks (GRNs). The key and original aspect of the software will be a reverse engineering feature which can change an existing circuit or create a new one to conform to a given list of constraints and features (such as constraining the concentration of a particular chemical to a range of values). This should be a useful feature for researchers and save time spent on manually designing a circuit with desired features.
	
		% Challenge?
	%- Quite a bit of research into Syn Bio -> Relatively new, so most resources are scientific papers.
	%- Syn Bio -> Simulation requires dealing with some maths, abstraction, algorithms.
	%- Rev. eng. requires quite a lot of maths
	%- Internal representation and program architecture: Need good design choices to make it maintainable.
	
	The challenge with this project is both theoretical and practical. Given that synthetic biology is not a field 
	% TODO !!!
	encountered by me before, the theoretical challenge is the significant research that goes into reading scientific papers to get an understanding of the biology that lies behind the computer simulations, which is ultimately necessary to be able to interpret more complex scientific papers which assume a level of familiarity with biology, as well as papers describing various computational modelling and simulation methods necessary for the project. The practical challenge is of implementing a complex system which aims to abstract and model a real life system. This requires good design choices to ensure a maintainable system is being developed. The final challenge is of being able to implement the key feature of the system: reverse engineering of networks. Implementing this feature requires mathematics and algorithmics skills to develop algorithms capable of converging into the correct network which satisfies all criteria.
	
	
	% TODO: Implementation part rather short? Could also include working with lbiraries and reading documentations, etc.
	
	\section{Background}
	% All projects should “come out of” something: previous related
	% knowledge/similar systems/choice of technology/experience
	% Show you are aware of this and are building on valuable previous work
	
	% Background:
	% - It is a common problem, apparentlym so:
	% - There are similar toolds:
	%- GenoCAD for design & sim using COPASI
	%- TinkerCell
	%- COPASI
	
	\par Given the mathematical basis of this project, it is appropriate to start with some background on biological models for gene networks. These allow researchers to represent the complex world of biology in the mathematical space. These models tend to try and capture changes occuring in the network over time. For example the ODE model describes the network as a series of ODEs (Ordinary Differential Equations) which model reaction rates such as transcription rate, degradation rate, etc\cite{sim_tutorial}. A stochastic model works in a similar way, again having reactions and reaction rates, however the key difference is that reaction rates are defined as functions in terms of the network state (the concentrations of each species in the network at a given time) and instead of representing reaction rates, these represent the propensity of the reaction occuring \cite{sim_tutorial}. Online databases such as the Biomodels Database \cite{biomodels} host reaction rates and other values derived from wet lab experiments which allow a simulation of these biological models over an arbitrarily long period of time.
	
	%This would be significantly more convenient than conducting wet lab experiments.
	\par There are a number of simulation algorithms. For example ODE simulation solves the system of ODEs to find the concentrations at a given time. Stochastic simulation using Gillespie algorithm\cite{gillespie_original} applies reactions randomly at randomly chosen intervals to simulate the network state over a given time period. This causes noises in the result but is more accurate and closer to how real biological systems work.
	
	\par These models and simulation algorithms can be used to create useful computer simulations. There exists computer simulation tools such as COPASI \cite{copasi} to simulate biological systems (including but not limited to GRNs) using a number of different simulation algorithms (including ODE and Gillespie). There are also many tools that go beyond merely doing simulation and offer CAD functions for design gene networks. Gene Designer \cite{gene_designer} offers many advanced features and focuses on detailed design of individual genes and their subcomponents (such as promoters, etc.). A similar tool, GenoCAD\cite{genocad}, also offers features to design the subcomponents of genes, but focuses more on features for designing genetic networks rather than single genes and also offers simulation using the COPASI API. A very similar tool is TinkerCell \cite{tinkercell}. 
	\par This project aims to design a similar software with the aforementioned design and simulation features but add a reverse engineering feature to ease the design of GRNs by automatically creating networks which have a set of desired features.
	
	\section{Progress}
	%Work completed (eg: coding started; interviews set up; framework for
	%comparison of algorithms developed)
	%A list can be helpful - but do not just give a list
	%Again, do not just say you have done something. Evidence?
	%Check out the example project reports - how (well) do they
	%achieve this?
	
	Most of the objectives mentioned in the specification document which should have been completed at this point in the project has been completed. However, the scope of some objectives had to be changed and new ones had to be added, which caused some parts of the timetable to be changed (see section \ref{timetable}).
	
	\subsection{Final report} 
	The original timetable stated that starting from week 3, the final report would be updated continuously as more features are added. This was eventually decided against as it is too early in the project and some details have not been clarified yet while some changes had to be and will likely be made to the original plan. Therefore, the writing of the final report will not start until term 2.
	
	\subsection{Testing} \label{progress-testing}
	The specification specified a unit test would be implemented for each objective. This could not be done because the components making up the code are too large and thus not easily testable. In the next stage of the project, these components will be broken down into smaller subcomponents and make use of shorter, more general functions to allow easy testing.
	
	
	\subsection{Simulation model} \label{sim-model}
	As mentioned in the specification, due to being ahead of schedule with the development progress, it was decided to switch to a more accurate and complex stochastic simulation model, which led to the scope of some objectives having to be changed, as explained in the sections that follow.
	
	\subsection{Objective 1 (Models)}  
	The change in simulation model meant that the internal models used to represent a GRN had to be changed as well. The stochastic simulation algorithm used (Gillespie algorithm \cite{gillespie_original}) works by randomly firing reactions. To account for this, a reaction model was implemented using an abstract \verb|Reaction| class (that has two abstract functions for returning the reaction rate function and change vector), which has subclasses for more specific reactions such as a \verb|TranscriptionReaction| class which makes use of the regulatory relationships stored in the network class to calculate its reaction rate function. Reaction rate function dictates the propensity of the reaction and is defined in terms of the species found in the system of which the reaction is a part. Furthermore, the model representing the whole network has been changed to a class storing a dictionary of species (key for the species name, value for the concentration/number of species) in the system and reactions as well as regulatory relationships between species. This class is called \verb|Network|. Finally, the original plan modelled each gene as a collection consisting of a promoter and coding regions. This would allow the user to create or import promoters and coding regions and build genomes modularly using the available parts. As the network is now a list of reactions and species, genes are not modelled this way. In the new system, If a gene needs to have the properties of a particular promoter, the reaction rate function can be changed to the values belonging to the promoter in question. This simplification will reduce the complexity of the system and make it more flexible.
	
	\subsection{Objective 2 (Simulation \& Visualisation)} 
	%TODO: Could elaborate on the resources you used for learning to simulate!
	%TODO: Should I maybe give more technical details? For all progresses?
	The original plan of implementing an ODE simulation has been achieved. The internal representation can be translated into a set of ODEs, and with the help of NumPy\cite{numpy} and SciPy\cite{scipy}, these equations can be solved. However as mentioned in section \ref{sim-model}, it was later decided to switch to a stochastic simulation. To perform the simulation, the Gillespie algorithm \cite{gillespie} was implemented in a \verb|GillespieSimulator| class accepting a \verb|Network| and a \verb|SimulationSettings| class (which has properties such as the simulation end time). The Gillespie algorithm is a widely used stochastic simulation algorithm but is also very computationally intensive. Roughly, the algorithm works by advancing the time of the simulation by a random amount (which can be very small if there are a large number of reactions) and by applying a randomly chosen (weighted using the reaction propensity) reaction. One method to improve the running time is called tau-leaping and might be implemented in the next part of the project. In its current state, the project is capable of performing stochastic simulations and visualising the results as a graph by using the \verb|GillespieSimulator| which also accepts a \verb|SimulationSettings| object to tell the visualiser which species to display.
	
	\subsection{Objective 5a (User can add new parts)} 
	As explained before, as the system now is a list of species and reactions (and not genomes consisting a promoter and coding regions), the user now has to add reactions and supply reaction rates, rather than parts.
	% TODO: Maybe also add species?
	
	\subsection{Objective 6a (SBML parsing)} 
	This objective was originally planned for a later date. However, it was decided to be implemented earlier, so that SBML models available online (for example on the Biomodels Database \cite{biomodels}) could be easily imported into the software without having to manually enter values to allow convenient testing. The parsing of the SBML document was achieved using the \verb|libsbml| \cite{libsbml} library while the \verb|SbmlParser| class converts the parsed SBML document into the internal representation of a network used by the program. \verb|SbmlParser| first builds a symbol table out of the global parameters in the SBML document. These parameters are returned by libsbml as \verb|ASTNode| objects but the program evaluates these to a floating point number by doing a post-order traversal of the node. Subsequently, reactions are converted into a \verb|CustomSbmlReaction| object. This is a subclass of the aforementioned \verb|Reaction| class but additionally accepts a string representation of the reaction rate equation and the symbol table built earlier. This string representation is obtained by calling a libsbml function. The \verb|CustomSbmlReaction|'s reaction rate function then uses Python's \verb|eval()| function to evaluate, given the symbol table passed to the class to substitute any symbols encountered, the string representation of the reaction rate equation. Using polymorphism, the \verb|GillespieSimulator| can call the \verb|CustomSbmlReaction| object's reaction rate function to get the new network state without any extra steps which might have been needed due to the reaction rate function being represented differently as a string for this reaction.
	
	\subsection{Evaluation of the progress}
	\par Evaluation of the models and the simulation are linked in the sense that if simulation can be proven to work, this would indicate a successful implementation of the model. The simulation is stochastic, meaning that the simulation results will be different each time, which complicates evaluation. However, the average of the results will reveal the same pattern. Therefore, the Repressilator \cite{repressilator} model was implemented using the program and simulated and visualised. The resulting graph showing the concentrations of proteins over time showed oscillating amounts of the proteins, which reflects the expected results as described in the original paper describing the Repressilator.
	\par Evaluation of SBML parsing involved downloading the Repressilator SBML file from the Biomodels Database, import it and simulate it. The simulation result reflected the expected results in the original paper, and thus the SBML parsing was deemed correct.
	
	\section{Choice of Methods and Tools}
	%Again, varies according to project type
	%- Development methods
	%- Technologies and languages
	%- Platforms, frameworks, datasets etc
	%- Project methodology (how you go about doing it)
	%- Data gathering methods
	%How are you doing things and what are you using to do these?
	
	%Progress:
	%- Models	-> Changed into reaction model
	%- Simulation & Visualisation	-> Changed to stochastic (as mentioned in spec)
	%- Half of SBML -> Why?
	%- How to prove they are done? Repressilator
	%- For each, say: Changes to original place, and current state.
	%- If not done, say why. With testing, not done because need to break it down further.
	
	\begin{itemize}
		\item Python is used as the main development language. This was chosen as it is a very flexible language with an extensive range of mathematical and scientific libraries.
		\item The library \verb|matplotlib| \cite{matplotlib} was used for visualisation of simulation results. This is a popular library and thus has been thoroughly tested by many other users. The library is easy to use and offers many useful features out-of-the-box such as zooming into produced graphs, panning around, scaling, etc.
		\item For parsing SBML files, the \verb|libsbml| \cite{libsbml} library was used. This library was developed by the original creators of the SBML standard and has excellent documentation, making it easy to work with.
		\item For designing the GUI, the \verb|GTK+ 3| \cite{gtk} library will be used, or more specifically its Python binding \verb|PyGObject| \cite{gtk_python}. This was chosen over the more popular \verb|tkinter| \cite{tkinter} library as \verb|GTK+ 3| has a more modern design and good documentation.
		\item For unit testing, the \verb|PyUnit| \cite{pyunit} library will be used. It is part of the xUnit framework and has very good documentation.
		\item The current state of the software does not have a dependency on \verb|NumPy/SciPy| \cite{numpy, scipy} as was mentioned in the specification.
		\item An integrated development environment (IDE) called \verb|PyCharm| \cite{pycharm} is being used.	
\end{itemize}
	
	
	
	\section{Project Management}
	\begin{itemize}
		%TODO: ?
		\item Most of the important milestones mentioned in the original timetable have been achieved at this point and steady progress have been made without many difficulties or problems.
	\end{itemize}
	
	\subsection{Updated Timetable} \label{timetable}
	
		The experience gained during this first part of the project has lead to some changes having to be made to the original timetable:
	\begin{itemize}
		% TODO: Help with testing the system, good, but, elaborate?
		
		
		\item A new task has been added (Basic UI functions) as it was realised the current tasks did not account for the UI design for functions such as opening files, adding reactions, etc.
		
		\item The tasks related to the reverse engineering feature have been given more time. The extra time has been accounted for by parallelising some tasks. Doing so will be possible for the first time parallelisation has been done as this time period coincides with the Christmas holiday, during which some other commitments such as attending lectures, etc. will not continue, thus more time can be invested in the project. For the second time, it will be possible as the parallelised tasks are the less time-consuming tasks.
		\item It has also been decided that a 7 day period will be allocated to researching and thinking about the type of approach to take for achieving reverse engineering and decide on the scope of the feature (i.e. which constraints will be offered).
		\item A new task called "Infrastructure" has been added, during which the current system will be extended to build the infrastructure necessary to handle reverse engineering functions.
		\item Another reverse engineering task called "performance improvements" have been added. It was anticipated that given the computational intensiveness of the Gillespie algorithm, a reverse engineering feature would take a very long time to run. Thus, this task will involve reducing the average running time of the Gillespie algorithm and the reverse engineering feature.
		\item As mentioned in section \ref{progress-testing}, it was not possible to implement unit tests. To solve this problem, a 1 week period will be allocated to refactoring the current code to break it up into smaller, easily testable functions and write unit tests for each. The lessons learned from this experience will be used in the future to avoid the same problem with unit tests.
	\end{itemize}
	
	\includegraphics[height=180pt]{timetable-new}
	
	%% Updated timetable:
	%- Finally, the updated timetable: GUI and rev. eng. at the same time.
	%- Add a 1 week period for thinking about constraints.
	
	% Tech used:
	%- Python: Quick, elegant, and also very flexible. Proved very useful when passing around functions
	%- Matplotlib for visualisation -> Does everything out of the box, popular, stable
	%- Libsbml for parsing, but requires an adapter to convert into internal rep.
	%- GTK+ -> More modern than the de-facto standard of tkinter (Usability?)
	%- No more dependence on NumPy/SciPy for simulation/visualisation
	%- Needs some more refactoring to allow unit testing
	%- PyUnit for unit testing.
	
	% Project management:
	%- Scrum?
	%- Allowing more time than needed worked: I am now on track
	
	
	
	
	\newpage
	
	\appendix
	\appendixpage
	\addappheadtotoc
	
	
	
	
	
	
	
	
	
	
	
	
	
	
	% Include your specification here!

	\section{Problem Statement}
	
	\par Synthetic Biology is a relatively new and rapidly growing field focusing on biological constructs not found in nature. One specific area of focus is Genetic Regulatory Networks (GRNs), where the aim of researchers is to create useful genetic circuits capable of producing desired amounts of proteins, such as the Repressilator \cite{repressilator}, a novel circuit which produces three proteins at amounts which oscillate with time.
	% This has useful applications such as 
	\par Since Synthetic Biology deals with new circuits, there tends to be a design stage before the network is put together in a real organism. Computer Aided Design (CAD) can help accomplish this task.
	\par Some existing CAD and simulation software include COPASI\cite{copasi}, a tool for simulating biologcal reactions (not only related to GRNs). Another tool, GenoCAD\cite{genocad}, is a web-based software allowing design of genetic circuits from a database of parts, export it to some popular formats (such as SBML), and simulate the design using COPASI's API. TinkerCell\cite{tinkercell} is a similar desktop-based software, allowing drag-and-drop gene circuit design and simulation.
	\par This project will include features to design and simulate gene circuits, just as existing software do. One feature not offered by these software is reverse engineering of circuits, which this project will implement. This feature will allow the user to specify a set of constraints and properties (such as the required concentration of a certain product) and create a gene circuit having all the properties and obeying all the constraints. This has the potential of cutting down on the design time for a circuit and help researchers.
	
	
	% - (Some possible extensions:
	% - IDE functionality
	% - Calculation of values from sequences)
	
	
	% (Extra constraints could include: Do it in this organism, do it using only these parts, do it using so many parts, etc.)
	
	\section{Objectives}
	\par The software will let the user choose biological parts from a catalogue (which includes user generated parts as well as those fetched from a host of public databases), combine them (similar to building an electric circuit) and set the details of the relationship between them (such as the equations describing their regulatory relation). Software will then be able to simulate the circuit to allow observing the concentration of products over time. The key reverse engineering feature of the project will allow the user to state the desired amount of products and let the software reverse engineer a circuit capable of producing them. These objectives can be expressed more specifically as a list of main tasks, and a set of subtasks that needs to be completed for the main task to be complete:
	
	
	\begin{enumerate}
		\item Should allow adding parts (as well as edit and delete them) to the network and specify their regulatory relations. Should allow the user to specify necessary network parameters.
		\begin{enumerate}
			\item Software should have a programmatic model of every supported part and its various parameters. This also includes non-trivial substances (which are not proteins or mRNA) which affect regulation, such lactose, as well as operons and other constructs which may be of importance to the process of regulation.
			\item Should keep track of the relations between each part (e.g. Which gene regulates a specific gene using which equation).
			\item Should store a programmatic model of the whole network, including parts, relations and global values. This subtask will build the foundation for the programmatic architecture of the project.
		\end{enumerate}
		
		\item Must be able to simulate the circuit.
		\begin{enumerate}
			\item Must be able to convert the programmatic model of the network into a series of mathematical equations which can be fed into an equation solver.
			\item Must solve the created equations and produce results (concentrations of products) at given intervals over a given period of time.
			\item Must be able to visualise the results of the simulation (e.g. using graphs).
		\end{enumerate}
		
		\item Must allow the user to specify a set of desired properties for the circuit to have (e.g. The amount produced of a certain protein should be above a certain number and the circuit should only use a given set of parts) and subsequently build a network which have these features.
		\begin{enumerate}
			\item Must convert the constraints given by the user into a set of mathematical constraints which the internal logic can work with.
			\item Needs to narrow down all possible circuit combinations to find the right one(s). 
		\end{enumerate}
		
		\item Must offer a user friendly UI.
		\begin{enumerate}
			\item Must visualise the circuit and its parts on the screen.
			\item Must allow drag-and-drop style interaction when adding and moving parts and defining relations between them.
			\item Must offer easy and intuitive access to parameters and values (such as clicking on a gene to access its associated parameters).
		\end{enumerate}
		
		\item It should be able to fetch biological parts (such as promoters, coding regions, etc.) and models (such as the Repressilator) from public databases (such as the BioModels\cite{biomodels} database, Registry of Standard Biological Parts \cite{rsbp}). Also needs to allow manuallly adding parts.
		\begin{enumerate}
			\item For some databases, needs to be able to scrape webpages.
			\item For some databases, needs to download machine-readable files and parse them.
			\item Needs to allow creation of parts from values entered by the user.
		\end{enumerate}
		
		\item Can import parts and models from a host of popular formats (such as SBML) and export to them.
		\begin{enumerate}
			\item Needs to be able to parse these formats.
			\item Needs to be able to convert the internal model used by this project to other popular formats.
		\end{enumerate}
		
	\end{enumerate}
	
	\subsection{Accuracy of the model}
	\par To be able to computationally represent a real Gene Regulatory Network (GRN), the software will have to adopt a mathematical model for GRNs. Researches have come up with various models of varying accuracy, ranging from boolean networks (which abstract some regulatory parameters) 
	
	to more accurate stochastic models (which aim to be accurate and take random noise into account as well).
	
	Furthermore, within each model, it is possible to abstract some factors which would have some effect on the process in real life. I am going to use Ordinary Differential Equations (ODEs) 
	
	to model the change in the concentration of mRNAs and proteins and model the gene regulation using Hill equation
	
	and take Hill coefficient to be 1. Finally, I will be considering the transcription rate, translation rate and mRNA and protein degradation rate for regulating transcription.
	
	\par However, it is possible that my chosen model may be too detailed and the time needed to implement it may cause me to miss some important milestones. On the contrary, it may be too abstract and not challenging enough. To avoid either problems, I have chosen a number of important dates (See the timetable section) on which I will consider my progress and decide whether I should revise my chosen model to make it more or less detailed.
	
	
	\section{Methods}
	
	\subsection{Software Methodology}
	
	\par I will be using a Scrum methodology, and in each sprint I will implement a new objective.
	
	
	\subsection{Version Management}
	\par For version management, I will be using Git, an open source software and will be pushing my changes to a remote private repository hosted at GitHub on my student account.
	
	\subsection{Evaluation \& Testing}
	\par I will be writing appropriate unit tests for each of the objectives apart from the ones related to a user friendly UI, as those cannot be tested with a unit test.
	\par To evaluate the circuit design and simulation features as a whole, I will build a network for which simulation results have been made available by other authors. The network will be using the same initial values, and the test will check whether my software produces the same results. 
	\par To evaluate the reverse engineering feature, I will input a set of constraints and let the software produce a circuit. Consequently, I will simulate the circuit in COPASI\cite{copasi}, an open source biological simulation software, and manually check whether the results obey the given constraints.
	
	\subsection{Timetable}
	
	\par I have prepared a Gantt chart to represent my plan for the project. I will be working on the final report all throughout the project, updating it as I add new features to the project. 
	
	\par During the "Biology Research" phase taking place at the beginning of the project, I will do some more reading on biology as I feel that improving my biology knowledge a little more will help with the project. The task names on the Gantt chart correspond to the ones listed in the Objectives section. 
	
	\par I planned for an "Extensions \& improvements" phase at the end. In the projects that I have done in the past, it usually was the case that after all the objectives were done, there were still some small bugs that needed fixing, as well as room for improvement and need for some polishing. Therefore, I wanted to leave some time at the end to make sure everything is working as expected. Furthermore, if things go as expected, I will also spend some time on implementing extensions which I didn't plan for due to time constraints.
	
	\par I purposefuly overestimated the time required for most tasks, as I know from my previous experiences that my estimates may sometimes be slightly inaccurate.
	
	\par Finally, as mentioned before, I plan to review my progress and revise the complexity of my biological model accordingly. I will do so on Week 9 of term 1, Week 3 of Term 2 and Week 8 of Term 2.
	
	\includegraphics[height=120pt]{timetable}
	
	\section{Resources}
	\begin{enumerate}
		\item The software will be written in Python.
		\item The simulation feature will require libraries capable of handling mathematical operations. For this, I will be using NumPy \cite{numpy} and SciPy \cite{scipy}.
		\item For the visualisation of the simulation results, a graph plotting library will be required. For this, I will use matplotlib \cite{matplotlib}.
		\item Git \cite{git} and GitHub \cite{github} will be used for version control and backing up the software.
		\item The data required (for biological parts and models) will be fetched from public databases such as the BioModels \cite{biomodels} and the Registry of Standard Biological Parts. \cite{rsbp}
	\end{enumerate}
	
	\section{Risks}
	\paragraph{IT failure} I will be regularly pushing my local commits on Git to GitHub, a remote repository. Therefore, if my personal computer fails I can carry on working using departmental computers.
	\paragraph{Underestimation of the time required} In the case that I have not been able to complete some required tasks on time, leading to the project getting derailed, I will focus on the most essential features and not implement some non-essential features, such as having a user friendly UI.
	\paragraph{Unexpected problem preventing me from working for a period of time} To avoid such problems, I have overestimated the time required to complete tasks in my timetable.
	\paragraph{Working on assignments for other modules (especially when there will be many at the same time) may cause me to miss planned completion deadlines for this project's tasks} This may especially be a problem since the project will be running for a long time, which may cause me to lose focus and spend a disproportionate amount of time on other assignments. I have found that breaking down tasks into SMARTly defined subtasks help with keeping my focus, as I will know exactly what needs to be done.
	
	
	
	\section{Ethical Considerations}
	\begin{itemize}
		\item All software and services I plan to use (except GitHub) are free and open source and their licences do not restrict usages for a university dissertation project.
		\item GitHub normally charges a monthly fee for a private repository. However, as I have a student account, I am able to freely use a private repository.
		\item All data required for the software is publicly available.
	\end{itemize}
	
	\bibliographystyle{plain}
	\bibliography{progress}
\end{document}