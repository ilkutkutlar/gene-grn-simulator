\documentclass{article}
\title{Modelling and Simulation of Gene Regulatory Networks}
\author{Ilkut Kutlar - 1621364}
\date{October 2018}

\begin{document}
	\maketitle
	
	\section{Problem Statement}
	
	\par Synthetic Biology is a relatively new and rapidly growing field [CITE] focusing on biological constructs not found in nature. One specific area of focus is Genetic Regulatory Networks, where the aim of researchers is to create useful genetic circuits capable of producing desired amounts of proteins, such as the Repressilator [CITE], a novel circuit which produces three proteins at amounts which oscillate with time.
	% This has useful applications such as 
	\par Since Synthetic Biology deals with new circuits, there needs to be a design stage before the network is put together in a real organism. Computer Aided Design (CAD) can help accomplish this goal.
	\par Some existing CAD and simulation software include COPASI[CITE], a general tool for simulating biologcal reactions given required parameters. Another tool, GenoCAD[CITE], is a web-based software allowing design of genetic circuits from a database of parts, export it to some popular formats (such as SBML), and simulate the design using COPASI. TinkerCell[CITE] is a similar desktop-based software, allowing drag-and-drop gene circuit design and simulation.
	\par One feature not offered by these software is reverse engineering of circuits. My software will include the design and simulation of gene circuits like existing tools. The key and original feature will allow the user to specify a collection of constraints and properties (such as the required concentration of a certain product) and create a gene circuit having all the properties and obeying all the constraints. This can save time for researchers.
	
	% - (Some possible extensions:
	% - IDE functionality
	% - Calculation of values from sequences)
	
	%TODO: Why is this feature so useful? Elaborate more on this
	% (Extra constraints could include: Do it in this organism, do it using only these parts, do it using so many parts, etc.)
	
	\section{Objectives}
	
\end{document}