\documentclass{article}
\usepackage[margin=1in]{geometry}
\author{Ilkut Kutlar - u1621364}
\title{CS310 - Progress Report}
\begin{document}
	\maketitle
	
	\section{Introduction}
	
	Synthetic Biology is a relatively new field concerned with creating new biological constructs not found in the nature with the aim of serving a useful purpose. The field involves an engineering process and therefore a researcher may need to go through a design stage before actually implementing the construct in a real organism. This project aims to make a CAD and simulation software for Gene Regulatory Networks (GRNs). The key and original ascpect of the software will be a reverse engineering feature which can change an existing circuit or create a new one to conform to a given list of constraints. This should be a useful feature for researchers and save time spent on manually designing a circuit with desired features.
	
	\section{Background}
	
	The CAD of GRNs is useful for researchers, and thus this is a common problem. Some other similar tools include GenoCAD which allows design of GRNs
	
	...
	% TODO: Complete this
	
	
	% Background:
	% - It is a common problem, apparentlym so:
	% - There are similar toolds:
	%- GenoCAD for design & sim using COPASI
	%- TinkerCell
	%- COPASI
	
	\section{Progress}
	Out of all the objectives mentioned in the specification document and which should have been completed at this point, most have been completed while some components had to be changed and some parts of the timetable changed.
	
	% Improvements:
	
	%Work completed (eg: coding started; interviews set up; framework for
	%comparison of algorithms developed)
	%A list can be helpful - but do not just give a list
	%Again, do not just say you have done something. Evidence?
	%Check out the example project reports - how (well) do they
	%achieve this?
	
	\paragraph{Objective 1 (Models)} As mentioned in the specification, due to being ahead of schedule with the development progress, it was decided to switch to a more accurate and complex stochastic model. Currently, all necessary models to build an internal representation of a GRN have been implemented.
	% TODO: MAybe also mention internal representation for stochastic?
	
	\paragraph{Objective 2 (Simulation \& Visualisation)} After switching from an ODE to stochastic model, the original plan to do an ODE simulation was also updated to do a stochastic simulation. The Gillespie algorithm used for stochastic simulation of the network have been implemented and can produce results. The visualisation of the produced results have also been implemented.
	
	\paragraph{Objective 6a (SBML parsing)} 
	
	\paragraph{Testing} The specification specified a unit test would be implemented for each objective. This has not been implemented yet because in their current states, the components making up the code are too large and thus not easily testable. In the next stage of the project, these components will be broken down into smaller subcomponents to allow easy testing.
	
	\subsection{Updated Timetable}
	
	\section{Design}		% Quality of design, does the system have a good design? This is the code stuff!
	
	\section{Choice of Methods and Tools}
	\begin{itemize}
		\item Python is used as the main development language. This was chosen as it is a very flexible language with an extensive range of mathematical and scientific libraries.
		\item The library \verb|matplotlib| was used for visualisation of simulation results. This is a popular library and thus has been thoroughly tested by many other users. The library is easy to use and offers many useful features out-of-the-box such as zooming into produced graphs, panning around, scaling, etc.
		\item For parsing SBML files, the \verb|libsbml| library was used. This library was developed by the original creators of the SBML standard and has excellent documentation, making it easy to work with.
		\item For designing the GUI, the \verb|GTK+ 3| library will be used. This was chosen over the more popular \verb|tkinter| library as \verb|GTK+ 3| has a more modern design and good documentation.
		\item For unit testing, the PyUnit library will be used.
		\item The current state of the software does not have a dependency on \verb|NumPy/SciPy| as was mentioned in the specification.
	\end{itemize}
	
	
	%Again, varies according to project type
	%- Development methods
	%- Technologies and languages
	%- Platforms, frameworks, datasets etc
	%- Project methodology (how you go about doing it)
	%- Data gathering methods
	%How are you doing things and what are you using to do these?
	
	%Progress:
	%- Models	-> Changed into reaction model
	%- Simulation & Visualisation	-> Changed to stochastic (as mentioned in spec)
	%- Half of SBML -> Why?
	%- How to prove they are done? Repressilator
	%- For each, say: Changes to original place, and current state.
	%- If not done, say why. With testing, not done because need to break it down further.
	
	%% Updated timetable:
	%- Finally, the updated timetable: GUI and rev. eng. at the same time.
	%- Add a 1 week period for thinking about constraints.
	
	% Tech used:
	%- Python: Quick, elegant, and also very flexible. Proved very useful when passing around functions
	%- Matplotlib for visualisation -> Does everything out of the box, popular, stable
	%- Libsbml for parsing, but requires an adapter to convert into internal rep.
	%- GTK+ -> More modern than the de-facto standard of tkinter (Usability?)
	%- No more dependence on NumPy/SciPy for simulation/visualisation
	%- Needs some more refactoring to allow unit testing
	%- PyUnit for unit testing.
	
	% Challenge?
	%- Quite a bit of research into Syn Bio -> Relatively new, so most resources are scientific papers.
	%- Syn Bio -> Simulation requires dealing with some maths, abstraction, algorithms.
	%- Rev. eng. requires quite a lot of maths
	%- Internal representation and program architecture: Need good design choices to make it maintainable.
	
	% Project management:
	%- Scrum?
	%- Allowing more time than needed worked: I am now on track
	
	
	\appendix
	% Include your specification here!
\end{document}